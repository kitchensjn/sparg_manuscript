\section*{Abstract} 

Spatial patterns of genetic relatedness among samples reflect the past movements of their ancestors. Our ability to untangle this history has the potential to improve dramatically given that we can now infer the ultimate description of genetic relatedness, an ancestral recombination graph (ARG). By extending spatial theory previously applied to trees, we generalize the common model of Brownian motion to full ARGs, thereby accounting for correlations in trees along a chromosome while efficiently computing likelihood-based estimates of dispersal rate and genetic ancestor locations, and associated uncertainties. We evaluate this model's ability to reconstruct spatial histories using individual-based simulations and unfortunately find a clear bias in the estimates of dispersal rate and ancestor locations. We investigate the causes of this bias, pinpointing a discrepancy between the model and the true spatial process at recombination events. This highlights a key hurdle in extending the ubiquitous and analytically-tractable model of Brownian motion from trees to ARGs, which otherwise has the potential to provide an efficient method for spatial inference, with uncertainties, using all the information available in the full ARG.


% highlighting a discrepancy between the histories of recombinant lineages and Brownian dispersal models. We investigate the underlying causes of these biases.


%We identify potential resolutions to this problem based on relaxing the constraints that ARGs place on the movement of lineages and show that ARG-based spatial inference can be used to effectively track the geographic history of admixed individuals. Approaches like this will be key to understanding the interplay of migration, recombination, drift, and adaptation in geographically spread populations.

\textbf{Keywords}: Ancestral recombination graph, spatial population genetics, population genetic inference, Brownian motion, networks, genetic ancestry.  

%Present-day spatial patterns of genetic relatedness reflect the past movements of our ancestors. In recombining organisms, each chromosome is a mosaic of genetic lineages brought together through recombination; this genealogy can be represented as a network called an ancestral recombination graph, or ARG. We first extend the Brownian dispersal model, previously restricted to tree-like genealogies, to more general network genealogies, such as ARGs. In this way, we can account for correlations along a chromosome while computing the maximum likelihood estimate of dispersal rate and the location of every genetic ancestor within an ARG. Further, we develop an efficient algorithm that allows for the application of the method to complex ARGs with a large number of samples. We evaluate our method's ability to reconstruct the spatial histories of these simulated lineages. Despite using the fullest information available in the data, we find that spatial estimates using ARGs are biased, uncovering an incompatibility between the histories of recombinant lineages and Brownian dispersal models. We identify potential resolutions to this problem based on reducing the constraints that ARGs place on the movement of lineages and show that ARG-based spatial inference can be used to effectively track the geographic history of recombinant lineages and admixed individuals. Approaches like this will be key to understanding the interplay of migration, recombination, drift, and adaptation in geographically spread populations.
 
 %The movement of ancestors influences the spatial patterns of genetic relatedness that we see today. We can therefore hope to infer the spatial history of a sample from a model of movement and a metric of relatedness. The ultimate metric of genetic relatedness is the complete genealogy of the sample. It is now possible to infer this genealogy for a sample of genomes whose lineages have recombined in the past. This genealogy, an ancestral recombination graph (ARG), is a network rather than a tree, making it difficult to extend many existing approaches. Here, we extend a convenient model of movement in continuous space (Brownian motion) from trees to ARGs and develop an efficient algorithm to compute the maximum likelihood estimate of dispersal rate and the location of every genetic ancestor in the ARG. Using simulations, we evaluate the accuracy and precision of these estimates and compare to previous methods. We find that, while the uncertainty in the dispersal estimate behaves appropriately, the dispersal estimate itself increases with the length of the genome considered. We pinpoint the issue as excess constraint under our model and demonstrate two ways forward with a simple model of Brownian motion. In contrast, we find that the location estimates under this model have good accuracy and in addition have smaller confidence intervals. This can be used for...... (something about locating recombination nodes and applications) 