\section{Introduction}

Life moves -- offspring disperse, populations admix, and species' ranges shift. While most of these events go unnoticed in the moment, they leave spatial patterns in the genetic diversity of a sample. Though often faint, we can use such signals to gain insight into the spatial history of the samples' shared genetic ancestors. 

% Many approaches to spatial inference, derived from early discrete space models \citep{Malecot1948, kimura1964stepping}, assume individuals are partitioned into populations or demes \citep[e.g.,][]{rousset1997genetic}. 
%This is an effective approach that has been used to infer ancient human migration \citep{hubisz2020,Wong2023} and contemporary disease spread \citep{Ignatieva2022, Ciccozzi2019}.
% These methods are not appropriate for systems with spatially continuous distributions of individuals as they force an arbitrary discretization of space which can ultimately bias the interpretation of results (Coop 2022).

One broad set of approaches to infer spatial history divides sample genomes up into a small number of geographic regions and, using allele frequency \citep[e.g.,][]{excoffier2021fastsimcoal2} or gene trees \citep[e.g.,][]{muller2018mascot}, estimates the split times and rates of gene flow between these regions. While useful, these approaches require the \textit{a priori} grouping of samples and can obscure the fact that there can be population structure at many geographical scales. For example, in many species genetic differentiation builds up relatively smoothly with the geographic distance between samples. 
%One of the simpler approaches to infer the spatial history of a sample is to divide space into a few geographic regions and compare the sequences between each.
%For example, the rate of gene flow between two regions can be inferred from the standardized variance in allele frequency between them \citep[$F_{ST}$;][]{wright1951genetical} under the island model \citep{Wright1931} or more complex demographic models can be fit from allele frequency spectra \citep[e.g.,][]{excoffier2021fastsimcoal2} and gene trees \citep[e.g.,][]{muller2018mascot}.
%Some information about the locations of a sample's genetic ancestors can also be inferred by asking what percent of its genome is most similar to the pool of reference genomes in each region \citep[e.g.,][]{23andme2022}. This leads to statements such as ``you have 50\% European ancestry", which suggests that half of your genome was inherited from ancestors that lived in Europe. While this approach does provide some spatial information, there are many issues \citep{coop2022genetic}, including the choice of the regions and the imprecise time and location of ancestors.
Such patterns motivate modeling approaches that treat space explicitly in so-called isolation-by-distance models.
A subset of these models assume local migration between groups of individuals (demes) on a lattice \citep[e.g. the stepping-stone ][]{Malecot1948, kimura1964stepping}, allowing the rate of increase in genetic differentiation ($F_{ST}$) with geographic distance to be used to estimate dispersal rates \citep{rousset1997genetic,rousset2000genetic}. The alternative is to treat space in a truly continuous manner, avoiding the need to assume well-mixed demes or a fixed layout of individuals.
% The classic models assume that individuals are uniformly distributed over space and that offspring disperse a normal deviate from their parent \citep{Wright1943,Malecot1948}. 
At its core, the classic continuum model \citep{Wright1943,Malecot1948} assumes lineages move as independent Brownian motions.
The tractability of this model makes it useful for spatial inference, for example, inferring dispersal rate and the locations of genetic ancestors from gene trees \citep[e.g.,][]{lemmon2008likelihood,Novembre2009}.
As a result, independent Brownian motion remains a common feature of many phylogeographic methods \citep[e.g.,][]{dellicour2021relax}.
Further research has aimed at resolving the issues that arise when assuming lineages move as independent Brownian motions, namely the clustering of individuals in forward-in-time models \citep{Felsenstein1975} and sampling inconsistency in backward-in-time models \citep{barton2010new}. The spatial $\Lambda$-Fleming-Viot process \citep{barton2010new,Barton2013,barton2010newEvol} is an alternative that, among other things, incorporates local density dependence to avoid these issues. However, the model's mathematical complexity makes inference computationally expensive \citep{Wirtz2023} and therefore limited to small sample sizes. Thus, independent Brownian motion, despite its limitations, continues to be an analytically tractable and computationally feasible model that is often useful for spatial inference, at least when dealing with non-recombining sequences.

On the empirical front, the increasing feasibility of whole-genome sequencing has led to an influx of genetic data and motivated advances in the inference of the genealogical history of a sample undergoing recombination \citep[e.g.,][]{Rasmussen2014,speidel2019method,kelleher2019inferring,schaefer2021ancestral,Wohns2022,Deng2024}. Recombination allows different regions of the same chromosome to have different gene trees. Although single-tree approaches may be suitable when studying non-recombining sequences (e.g., mitochondrial DNA), they do not capture the range of genetic relationships found across recombining genomes nor the correlations between these relationships. For this, we must turn to the ancestral recombination graph (ARG).

%An ARG contains the complete genetic history of a sample of recombining genomes \citep{Hudson1983, Griffiths1996, lewanski2023era}. It is commonly displayed as 1) a sequence of trees with each tree representing the history of a continuous block of the genome or 2) a single directed acyclic graph with annotated edges corresponding to their genomic intervals (see Fig \ref{fig:SingleLineage}A for an example of each). While the two representations can be interchangeable, tree sequences often lack the recombination events that tie the trees together and are further simplified, removing this equivalency \citep{Wong2023}. ARGs are an incredibly rich source of information about the history of the sample \citep{harris2019database,hejase2020summary,lewanski2023era}. 

%The utility of ARGs for spatial inference is still in its nascent stages, but growing. A number of approaches now exist for implicit space \citep[e.g.,][]{Guo2022,fan2023likelihood}. Three approaches also exist for explicit continuous space: \cite{Osmond2021} infer dispersal rate and ancestral locations under Brownian motion independently applied to trees sparsely sampled from a simplified ARG; \cite{Wohns2022} infer ancestral locations by placing each node at the midpoint of its descendant nodes on a simplified ARG; and \cite{grundler2024} infer dispersal rate and ancestral locations by placing each node at the location that minimizes a migration cost averaged over the trees in a simplified ARG that the node appears in. None of these continuous space approaches utilize all of the information contained in the ARG with an explicit model for spatial movement, which is what we aim to do here. 
% \cite{grundler2024} use all the trees in a tree sequence by using a migration cost minimization approach. However, while the cost function is motivated by Brownian motion like movement (in fact for a single tree, the method is identical to maximum likelihood estimates under the Brownian motion), they do not use an explicit spatial model for their estimates. Further, the cost function for each node is computed independently of the cost function for other nodes, therefore artificially decoupling the location estimates for different nodes. 
 
An ARG encodes the complete genetic history of a sample of recombining genomes \citep{Hudson1983, Griffiths1996, lewanski2023era}. It has proven to be an incredibly rich source of information about the history of the sample  \citep{harris2019database,hejase2020summary,lewanski2023era}. 
% including spatial reconstruction \citep{Osmond2024,Wohns2022,grundler2024}. 
One particularly promising application is spatial inference.
Two recent methods use an ARG to provide point estimates of ancestor locations \citep{Wohns2022,grundler2024} while another provides the full probability distribution for an ancestor's location but ignores correlations between trees \citep{osmond2024estimating}.
Here, we extend the classic Brownian motion model for trees to describe movement down an ARG. Under this model, we derive the full likelihood of the sample locations given an ARG, allowing us to infer the dispersal rate and the probability distribution for the location of every genetic ancestor in the ARG using the complete genealogical history of the sample. In doing so, we highlight a key problem in extending a classic model of movement from trees to ARGs.


%in applying Brownian motion to ARGs arising from the loopy structure of ARGs and the unlikeliness of two independent Brownian motions meeting at a recombination node.



%both the mathematical and computational challenges that are posed due to recombination loops in the ARG. We then provide a mathematically rigorous solution and a computationally fast algorithm to infer spatial histories, which we test with simulations. While using all the information in an ARG, it is not immediately clear how well our method will infer spatial history, especially given the unlikeliness of two independent Brownian motions meeting \citep{Etheridge2019}. 
% Throughout, we emphasize the utility of Brownian motion for spatial inference with ARGs and suggest modifications where it falls short.



